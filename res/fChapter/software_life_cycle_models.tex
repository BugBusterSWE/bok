\chapter{Modelli di cicli di vita del software}

\section{Modello linare}
Nel modello lineare le fasi di sviluppo del software sono compiute sequenzialmente con feedback e iterazioni al seguito di integrazione, test e distribuzione del singolo prodotto.
Spesso sono riferiti come modelli \textbf{predittivi}, in quanto gestiscono i requisiti sofware per le entit\'a possibilie, durante l'inizio e la pianificazione.
I requisiti sono spesso rigorosamente controllati. Un loro cambio \'e dovuto a richieste di cambiamento, processati da una \textbf{change control board}. [SWEBok - 8.2.2]

\section{Modello iterativo}
Nel modello iterativo il software viene sviluppato in incrementi di funzionalit\'a su cicli iterativi.[SWEBok - 8.2.2]

\section{Modello agile}
Il modello agile implica dimostrazioni frequenti sul software funzionante al cliente o ad un utente reppresentativoche direttamente sviluppa in corti cicli iterativi producendo piccoli incrementi del software atteso.
Insieme al modello iterativo, il modello agile viene anche definito modello \textbf{adattivo} per l'uso di revisioni successive. In questo modo i requisiti diventano flessibili in base all'evoluzione del prodotto software.
Viene anche definito a \textbf{scopo produttivo} e ad \textbf{caratteristiche iniziali di alto livello}; anche se comunque designate per la facilit\'a d'evoluzione dei requisiti software durante il progetto. [SWEBok - 8.2.2]
