\chapter{Misurazioni software}
\section{Principio}
Prima di introdurre un nuovo processo o modificare quello attuale, sarebbe da ottenere i risulati delle misurazioni per provvedere una \gls{baseline} per \textbf{comparare} la situazione corrente con quella nuova.

\section{Processi software e misurazione del prodotto}
Il processo software e la misurazione del prodotto sono interessati a determinare l'\gls{efficacia} e l'\gls{efficienza} di un \textbf{processo}, \textbf{attivit\`a} o \textbf{compito}. \newline
Lo sforzo \'e la misura principale ed misurato in \textbf{unit\`a}:
\begin{itemize}
\item Ore/persona
\item Mese/persona
\item Ore/staff
\item Mese/staff
\end{itemize}
o l'equivalente del costo unitario, in euro o dollaro. \newline
Le misurazioni di un processo software efficacie devono essere misurazioni di attributi \textbf{rilevanti} del prodotto; ad esempio, la misurazione di difetti scoperti e corretti del software durante la fase di test. \newline
I concetti di efficacia ed efficienza sono \textbf{indipendenti}. Un processo pu\`o essere efficacie ma non efficiente, e viceversa. Le cuase di bassa efficacia e/o efficienza pu\`o essere:
\begin{itemize}
\item Inadeguati prodotti in input
\item \textbf{Inesperienza del personale}
\item Mancanza di un'infrastruttura e strumenti adeguati
\item \textbf{Imparare nuovi processi}
\item Prodotti complessi
\item \textbf{Dominio del prodotto non famigliare}
\end{itemize}

