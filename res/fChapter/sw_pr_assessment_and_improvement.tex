\chapter{Processi software di stima e miglioramento}
\section{Processi di stima}
\subsection{Principi}
I processi di stima del software sono usati per \textbf{valutare} la forma e il contenuto di un processo software, usando un set \textbf{standardizzato} di criteri. \newline
La valutazione, perfomata dall'acquirente o da un agente incaricato, \'e usata come \textbf{indicatore} se i processi usati dal fornitore sono accettabili per l'acquirente.\newline
In pi\'u, possono includere problemi come valutare se i criteria di entrata e uscita del software sono in fase di raggiungimento ( PRE e POST ), per rivedere i fattori di rischio e di gestione, o per identificare le lezioni imparate.\newline

Da non confondere i processi di stima con i processi di verifica. Le verifiche snono condotte per \textbf{accertare} la conformit\'a con le politiche e gli standard. Inoltre, provvedono gestione della visibilit\'a nelle operazioni attualmente svolte in modo da prendere decisioni riguardo questioni che impattano:
\begin{itemize}
\item sviluppo del progetto
\item attivit\'a di manutenzione
\item software in relazione con l'argomento
\end{itemize}
Le stime invece sono attuate per determinare il livello di \textbf{capacit\`a} o \textbf{maturit\`a} e identificare i processi sofware da migliorare.

\subsection{Modelli dei processi di stima}
Tipicamente i modelli di stima includono criteri di stima la quale compongono buone pratiche. \newline
Queste pratiche posso essere indirizzate ai singoli processi o interi argomenti dello sviluppo sofware.

\subsection{Metodi dei processi di stima}
Un metodo per un processo di stima del sofware pu\`o essere \glslink{stimaqualitativa}{qualitativo} o \glslink{stimaquantitativa}{quantitativo}. \newline
L'obiettivo del processo di stima \`e aumentare la conoscenza che stabilir\`a lo stato corrente di uno o pi\`u processi e provvede un \textbf{fondamento} per il processo di migliormaneto.

\section{Processi di migliormaneto}
\subsection{Modelli dei processi di miglioramento}
I modelli di miglioramento dei processi enfatizzano i \textbf{cicli iterativi} di continui miglioramenti. \newline
Il modello \textbf{Plan-Do-Check-Act} \`e una buon approccio iterativo:
\paragraph{Planning}
Di \textbf{Plan}, sono le attivit\`a di miglioramento che includono l'identificazione e l'assegnazione di priorit\`a ai miglioramenti desiderati.
\paragraph{Doing}
Di \textbf{Do}, introdurre un miglioramento, includendo gestione dei cambiamenti e allenamenti.
\paragraph{Checking}
Di \textbf{Check}, valutazione del miglioramento, comparandolo a \textbf{precedenti} o \textbf{ideali} risultati e costi di processi.
\paragraph{Acting}
Di \textbf{Act}, fai ulteriori modifiche.

\subsection{Classificazione continua o a fasi dei processi software}
La valutazione delle \textbf{capacit\`a} e della \textbf{maturit\`a} del processo software avviene mediante cinque o sei livelli. L'assegnazione dei livelli pu\`o avvenire mediante classificazione \glslink{classificazionecontinua}{continua} o \glslink{classificazionefasi}{a fasi}. \pagebreak 
\begin{table}[h]
\centering
\begin{tabular}{|c|p{5cm}|p{5cm}|}
\hline
\multicolumn{3}{|c|}{Livelli di classificazione dei processi software} \\
\hline
Livello & Continua - Rappresentazione del livello di capacit\`a & A fasi - Rappresentazione del livello di maturit\`a \\ \hline
0 & Incompleta & \\ \hline
1 & Eseguita & Iniziale \\ \hline
2 & Gestita & Gestita \\ \hline
3 & Definita & Definita \\ \hline
4 & & Quantitivamente gestita \\ \hline
5 & & Ottimizzata \\
\hline
\end{tabular}
\caption{Tabella 8.1 dello SWEBok 8.3.4}
\label{tab:swe8_1}
\end{table}

\paragraph{Livello 0}
Indice che il software \`e incompleto o non puo essere eseguito.
\paragraph{Livello 1}
Il processo \`e stato eseguito ( capacit\`a ), o i processi al livello di maturit\'a 1 sono stati eseguiti ma su una base \textbf{informale} e \textbf{ad hoc}.
\paragraph{Livello 2}
Il processo ( capacit\`a ) o i processi al livello di maturit\'a 2 sono stati eseguiti in modo che forniscono una gestione della visibilit\`a dentro prodotti di lavoro intermedi e possono esercitare qualche controllo sulle transizioni tra processi. 
\paragraph{Livello 3}
I singoli processi o i processi di maturit\`a 3, pi\`u quelli di livello 2, sono \textbf{ben definiti} e sono \textbf{ripetuti} tra differenti progetti. \newline
Il livello 3 di capacit\`a o maturit\`a provvede il \textbf{fondamento} per l'uso di processi di miglioramento, in quanto gli altri processi sono condotti in modo \textbf{simile}. \newline
\paragraph{Livello 4}
Possono essere applicate misure quantitative e usate per i processi di stima.
\paragraph{Livello 5}
I meccanismi per continui processi di miglioramento sono applicati.
\newline \newline
Le reppresentazioni continue o a fasi possono essere usate per determinare l'\textbf{ordine} in qui i processi software sono da migliorare. \newline
Nella rappresentazione continua, livelli di capacit\`a differenti per diversi processi software provvedono una \textbf{linea guida} per determinare l'ordine dei processi da migliorare. \newline 
Nella rappresentazione a fasi, soddisfacendo gli obiettivi di un set di processi dentro un livello di maturit\`a si raggiunge quel livelo, gettando le fondamente per il raggiungimento del prossimo livello.  
