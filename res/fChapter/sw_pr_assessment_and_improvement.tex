\chapter{Processi software di stima e miglioramento}
\section{Processi di stima}
\subsection{Principi}
I processi di stima del software sono usati per \textbf{valutare} la forma e il contenuto di un processo software, usando un set \textbf{standardizzato} di criteri. \newline
La valutazione, perfomata dall'acquirente o da un agente incaricato, \'e usata come \textbf{indicatore} se i processi usati dal fornitore sono accettabili per l'acquirente.\newline
In pi\'u, possono includere problemi come valutare se i criteria di entrata e uscita del software sono in fase di raggiungimento ( PRE e POST ), per rivedere i fattori di rischio e di gestione, o per identificare le lezioni imparate.\newline

Da non confondere i processi di stima con i processi di verifica. Le verifiche snono condotte per \textbf{accertare} la conformit\'a con le politiche e gli standard. Inoltre, provvedono gestione della visibilit\'a nelle operazioni attualmente svolte in modo da prendere decisioni riguardo questioni che impattano:
\begin{itemize}
\item sviluppo del progetto
\item attivit\'a di manutenzione
\item software in relazione con l'argomento
\end{itemize}
Le stime invece sono attuate per determinare il livello di \textbf{capacit\`a} o \textbf{maturit\`a} e identificare i processi sofware da migliorare.

\subsection{Modelli dei processi di stima}
Tipicamente i modelli di stima includono criteri di stima la quale compongono buone pratiche. \newline
Queste pratiche posso essere indirizzate ai singoli processi o interi argomenti dello sviluppo sofware.

\subsection{Metodi dei processi di stima}
Un metodo per un processo di stima del sofware pu\`o essere \glslink{stimaqualitativa}{qualitativo} o \glslink{stimaquantitativa}{quantitativo}.

