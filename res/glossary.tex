\newglossaryentry{sistematico}
{
name=sistematico,
description={Colui che sa far quella cosa - darsi delle regole - approcciarsi in al problema con metodo. Ci\`o porta valore all'efficenza e all'efficacia}
}

\newglossaryentry{stakeholder}
{
name=stakeholder,
description={Dall'inglese, portatore di interessi. Sono l'insieme di persone a vario titolo coinvolte nel ciclo di vita del Software con influenza sul prodotto}
}

\newglossaryentry{processo}
{
name=processo, 
description={Multiple definizioni:
\begin{enumerate}
\item Detto in inglese come \textit{Way of working}. Esso \`e rappresentabile come un automa a stati, dove ogni stato rappresenta uno stadio del ciclo di vita del processo
\item [SWEBok 8-1] Un insieme di attività, interconnesse, che trasforma uno o più input in output consumando risorse.
\end{enumerate}
}
}

\newglossaryentry{procedura}
{
name=procedura,
description={Una procedura \'e un ordinato insieme di passi o, alternativamente, controlli del lavoro per eseguire il task}
}


\newglossaryentry{sync}
{
name=sync,
description={Dall'inglese ``Punto di fine", scarico da cui non esco pi\`u (ovvero uno stato finale)}
}

\newglossaryentry{source}
{
name=source,
description={Dall'inglese, stato che ammette solo stati in uscita}
}

\newglossaryentry{ciclovitasoftware}
{
name=ciclo di vita del software,
description={Sono gli stati che il prodotto assume dal concepimento al ritiro}
}

\newglossaryentry{efficacia}
{
name=efficacia,
description={Conformit\`a al contratto. Si garantisce cio\`e ci\`o che si deve fare, ed \`e determinata dal grado di conformit\`a del progetto rispetto alle norme vidgenti e agli obiettivi prefissati. L'efficacia \`e direttamente proporzionale alla quantit\`a di risorse impiegate}
}

\newglossaryentry{efficenza}
{
name=efficenza,
description={L'efficienza \`e la capacit\`a di azione o di produzione con il minimo di scarto, di spesa, di risorse e di tempo impiegati. \`E inversamente proporzionale alla aquantit\`a di risorse impiegate nell'esecuzione delle attivit\`a richieste}
}

\newglossaryentry{bestpratice}
{
name=best practice,
description={Prassi (modo di fare) che per esperienza e per studio abbia mostrato di garantire i miglio risultati in circostanze note e specifiche}
}

\newglossaryentry{iterazione}
{
name=iterazione,
description={Procedere per iterazioni significa operare raffinamenti o rivisitazioni. Essa \`e associabile a un'operazione, ad un qualcosa fatto prima (gi\`a fatto). Questa operazione \`e potenzialmente distruttiva e ha caratteristiche molto pericolose, perch\`e non sa garantire come finir\`a ed \`e una ripetizione di una cosa che ho gi\`a fatto},
plural=iterazioni
}

\newglossaryentry{incremento}
{
name=incremento,
description={Avvicinamento alla meta che si compie in due modi: aggiungendo o togliendo. Procedere per incrementi significa aggiungere a un impianto base. Un incremento non pu\`o mai tornare sui suoi passi, ed \`e preferibile rispetto alla iterazione, perch\`e pianifica i passi e ci\`o significa che si arriver\`a a una fine},
plural=incrementi
}

\newglossaryentry{prototipo}
{
name=prototipo,
description={Originario, abbozza, serve per capire se si sta andando in una direzione giusta o no. Esistono due tipi di prototipi: usa e getta da usare solamente se il beneficio \`e molto maggiore del costo per produrla, altrimenti se si presta ad essere la soluzione, anche se pu\`o essere una base per una iterazione},
plural=prototipi
}
\newglossaryentry{riuso}
{
name=riuso,
description={Due tipi di riuso: opportunistico (in stile copy-paste, a basso coso ma scarso impatto), altrimenti l'altro uso \`e quando si sa cosa prendo, so perch\`e lo prendo e so cosa fa. Fare software \`e fondalmentalmente riuso. \`e quindi una delle attivit\`a pi\`u importanti di SWE, assume una connotazione positiva},
plural=riusi
}

\newglossaryentry{controllodiversione}
{
name=controllo di versione,
description={--Da completare successivamente--}
}

\newglossaryentry{disciplinato}
{
name=disciplinato,
description={Saper prevedere i costi. Avere una quantit\`a credibile, seguendo le regole. Essere disciplinati significa anche seguire un ordine preciso degli stati nel ciclo di vita del software},
plural=disciplinati
}

\newglossaryentry{controllodeiprocessi}
{
name=controllo dei processi,
description={Luogo in cui si pongono delle regole per essere sempre efficaci e disciplinati}
}

\newglossaryentry{trigger}
{
name=trigger,
description={Evento che causa il cambiamento di arco nel ciclo di sviluppo del software. Attivit\`a la quale fa cambiare lo stato dell'automa}
}

\newglossaryentry{fase}
{
name=fase,
description={Durata temporale entro uno stato di ciclo di vita o in una transizione tra essi}
}


\newglossaryentry{pre-condizione-cascata}
{
name=pre-condizione,
description={Nel modello a cascata, la pre-condizione \`e ci\`o che \`e verificato prima di entrare in un certo stato}
}

\newglossaryentry{post-condizione-cascata}
{
name=post-condizione,
description={Nel modello a cascata, \`e ci\`o che dev'essere vero dopo lo svolgimento delle attivit\`a}
}

\newglossaryentry{meta-modello}
{
name=meta-modello,
description={Insieme di regole, vincoli e teorie utilizzate per la modellazione di una classe di problemi con astrazione dal mondo reale}
}

\newglossaryentry{casoduso}
{
name=caso d'uso,
description={Tecniche per individuare i requisiti funzionali. Queste Tecniche devono essere comprensibili anche all'utente committente. Il caso d'uso descrive l'insieme di funzionalit\`a del sistema come sono percepite dagli utenti}
}

\newglossaryentry{scenario}
{
name=scenario,
description={Rappresenta una sequenza di passi che descrivono iterazioni tra gli utenti e il sistema}
}

\newglossaryentry{attore}
{
name=attore,
description={Elemento esterno al sistema che interagisce con sistema}
}

\newglossaryentry{progetto}
{
name=progetto,
description={Insieme di tre elementi importanti: \begin{enumerate}
\item Insieme ordinato di compiti da svolgere
\item I compiti da svolgere sono pianificati da inizio a fine
\item I vincoli che vengono tenuti conto quando si pianifica nascono da quanto tempo ho a disposizione per l'intero progetto e quali strumenti \`e possibile utilizzare per dare i risultati attesi
\end{enumerate}
}
}

\newglossaryentry{rischio}
{
name=rischio,
description={\`e il non aver tenuto conto che le cose possono non andare come avevamo considerato}
}

\newglossaryentry{slack}
{
name=slack,
description={Margine tra inizio e fine di un'attivit\`a. In italiano \`e sinonimo di lasco}
}

\newglossaryentry{primaryprocess}
{
name = processo primario,
descritpion = {[def. SWEBok-v3 8-2.1.3] I processi primari includono processi sofware per: \begin{itemize}
\item Sviluppo
\item Operazioni o funzioni
\item Mantenimento
\end{itemize}
del software}
}

\newglossaryentry{supportinprocess}
{
name = processo di supporto,
description = {[def. SWEBok-v3 8-2.1.3] Processi di supporto sono applicati discontinuamente o continuamente durante il ciclo di vita del software a supporto dei processi primari; questi includono:\begin{itemize}
\item Gestione configurazione
\item Controllo della qualit\'a
\item Verifica e validazione
\end{itemize}
}
}

\newglossaryentry{organizationalprocess}
{
name = processo organizzativo,
description = {[def. SWEBok-v3 8-2.1.3] I processi organizzativi provvedono al supporto all'ingegneria del software. Includono: \begin{itemize}
\item Formazione
\item Analisi di misura del processo
\item Gestione dell'infrastruttura
\item Portfolio e riuso
\item Organizzazione miglioramento dei processi
\item Gestione del modello del ciclo di vita del software
\end{itemize}
}
}

\newglossaryentry{SDLC}
{
name = Ciclo di vita dello sviluppo software (SDLC),
descritpion = {[def. SWEBok-v3 8-2] Un ciclo di vita dello sviluppo software include i processi software usati per speficifare e transformare requisiti software in un prodotto software finito.
}

\newglossaryentry{SPLC}
{
name = Ciclo di vita del prodotto software (SPLC),
descritpion = {[def. SWEBok-v3 8-2] Un ciclo di vita del prodotto software include un SDLC pi\'u addizionali processi software che provvedono al: \begin{itemize}
\item distribuzione
\item mantenimento
\item supporto
\item evoluzione
\item ritiro
\end{itemize}
e tutti gli altri processi di inizio al ritiro, includendo processi di gestione per il controllo della configurazione e della qualit\'a applicati durante il ciclo di vita del prodotto software. }

\newglossaryentry{camminocritico}
{
name=cammino critico,
description={Sequenza di attivit\`a-progetto che ha lo slack pi\`u piccolo}
}
